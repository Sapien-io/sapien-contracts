\documentclass{article}
\usepackage{amsmath}
\usepackage{amssymb}
\usepackage{graphicx}
\usepackage{hyperref}
\usepackage{multirow}

\title{Sapien Staking Multiplier Model}
\author{Sapien AI}
\date{\today}

\begin{document}
\maketitle

\section{Multiplier Formula}

The Sapien staking multiplier is calculated using a multiplicative approach that combines both staking amount and lockup time into a unified bonus system:

\begin{equation}
M_{final} = M_{base} + \frac{x \times B_{max}}{10000}
\end{equation}

Where:
\begin{equation}
x = \frac{A_{norm} \times T_{norm}}{10000}
\end{equation}

And:
\begin{itemize}
    \item $M_{base} = 1.00$ (10000 basis points)
    \item $A_{norm}$ is the normalized amount (0 to 10000 basis points)
    \item $T_{norm}$ is the normalized time (0 to 10000 basis points)
    \item $B_{max} = 0.50$ (5000 basis points maximum bonus)
    \item $x$ is the combined effect of amount and time
    \item $M_{final}$ ranges from 1.00x to 1.50x (10000 to 15000 basis points)
\end{itemize}

\section{Normalization}

Both staking amount and lockup time are normalized to basis points (0-10000) based on their respective maximum values:

\subsection{Amount Normalization}
\begin{equation}
A_{norm} = \frac{A \times 10000}{A_{max}}
\end{equation}

Where:
\begin{itemize}
    \item $A$ is the staked amount in tokens (clamped to $A_{max}$ if exceeded)
    \item $A_{max} = 2500$ tokens
    \item $A_{norm}$ ranges from 0 to 10000 basis points
\end{itemize}

\subsection{Time Normalization}
\begin{equation}
T_{norm} = \frac{T \times 10000}{T_{max}}
\end{equation}

Where:
\begin{itemize}
    \item $T$ is the lockup period in seconds (clamped to $T_{max}$ if exceeded)
    \item $T_{max} = 365$ days = 31,536,000 seconds
    \item $T_{norm}$ ranges from 0 to 10000 basis points
\end{itemize}

\section{Combined Effect Calculation}

The key innovation in this model is the multiplicative combination of amount and time effects:

\begin{equation}
x = \frac{A_{norm} \times T_{norm}}{10000}
\end{equation}

This creates a synergistic effect where:
\begin{itemize}
    \item Small amounts with short lockups yield minimal bonus
    \item Large amounts with short lockups yield moderate bonus
    \item Small amounts with long lockups yield minimal bonus
    \item Large amounts with long lockups yield maximum bonus
\end{itemize}

\section{Bonus Calculation}

The final bonus is calculated as:

\begin{equation}
B_{final} = \frac{x \times B_{max}}{10000}
\end{equation}

Where:
\begin{itemize}
    \item $x$ is the combined effect (0 to 10000)
    \item $B_{max} = 5000$ basis points (0.50x)
    \item $B_{final}$ ranges from 0 to 5000 basis points
\end{itemize}

\section{Example Calculations}

\subsection{Minimum Stake, Minimum Time (1 token, 30 days)}
\begin{align*}
A_{norm} &= \frac{1 \times 10000}{2500} = 4 \\
T_{norm} &= \frac{30 \times 86400 \times 10000}{31536000} = 821 \\
x &= \frac{4 \times 821}{10000} = 0.33 \\
B_{final} &= \frac{0.33 \times 5000}{10000} = 0.16 \\
M_{final} &= 10000 + 0.16 = 10000.16 \approx 10000
\end{align*}

\subsection{Medium Stake, Medium Time (1000 tokens, 180 days)}
\begin{align*}
A_{norm} &= \frac{1000 \times 10000}{2500} = 4000 \\
T_{norm} &= \frac{180 \times 86400 \times 10000}{31536000} = 4932 \\
x &= \frac{4000 \times 4932}{10000} = 1973 \\
B_{final} &= \frac{1973 \times 5000}{10000} = 986 \\
M_{final} &= 10000 + 986 = 10986
\end{align*}

\subsection{Maximum Stake, Maximum Time (2500 tokens, 365 days)}
\begin{align*}
A_{norm} &= \frac{2500 \times 10000}{2500} = 10000 \\
T_{norm} &= \frac{365 \times 86400 \times 10000}{31536000} = 10000 \\
x &= \frac{10000 \times 10000}{10000} = 10000 \\
B_{final} &= \frac{10000 \times 5000}{10000} = 5000 \\
M_{final} &= 10000 + 5000 = 15000
\end{align*}

\section{Visualization}

\begin{figure}[ht]
    \centering
    \includegraphics[width=\textwidth]{multiplier_chart.png}
    \caption{Sapien Staking Multiplier Surface showing the relationship between staked amount and lockup time. The multiplicative nature creates a curved surface where the maximum 1.50x multiplier is only achievable through the combination of maximum stake (2500 tokens) and maximum lockup (365 days).}
    \label{fig:multiplier_surface}
\end{figure}

\section{Key Properties}

\begin{enumerate}
    \item \textbf{Base Reward}: All stakers receive a base 1.00x multiplier
    \item \textbf{Multiplicative Synergy}: Amount and time bonuses multiply together, creating stronger incentives for combining both
    \item \textbf{Progressive Scaling}: Higher multipliers require significant commitment in both dimensions
    \item \textbf{Maximum Reward}: The maximum 1.50x multiplier requires both maximum stake (2500 tokens) and maximum time (365 days)
    \item \textbf{Continuous Scaling}: The multiplier scales continuously based on the product of normalized inputs
    \item \textbf{Input Clamping}: Amounts above 2500 tokens and lockups above 365 days are clamped to maximum values
\end{enumerate}

\section{Implementation Notes}

The multiplier is implemented in Solidity using basis points (1.00x = 10000):
\begin{itemize}
    \item Base multiplier: 10000 basis points (1.00x)
    \item Maximum tokens: 2500 × $10^{18}$ wei
    \item Maximum lockup: 365 days = 31,536,000 seconds
    \item Maximum bonus: 5000 basis points (0.50x)
    \item Final multiplier range: 10000-15000 basis points (1.00x-1.50x)
\end{itemize}

The calculation ensures that:
\begin{itemize}
    \item Inputs are clamped to maximum values to prevent overflow
    \item All calculations use integer arithmetic to avoid floating-point precision issues
    \item The combined effect creates natural incentive alignment for long-term, high-value staking
\end{itemize}

\end{document}